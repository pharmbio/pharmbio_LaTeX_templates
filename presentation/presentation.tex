\documentclass[aspectratio=169]{beamer}

\usepackage{pgf}  
\usepackage{tikz}
\usetikzlibrary{arrows}
\usepgflibrary{shapes.arrows} 
\usetikzlibrary{intersections}
\usetikzlibrary{calc}
\usetikzlibrary{fit}
\usetikzlibrary{automata,positioning}
\usepackage{pgfplots,stackengine}
\usepackage{fontspec}
\usepackage{fancyvrb}
\usepackage{wasysym}
\usepackage{unicode-math}
\usepackage{import}
\usepackage{rotating}
\usepackage{gensymb}
\usepackage{chemfig}
\usepackage{rotating}
\usepackage{booktabs}
\usepackage{pifont}
\usepackage{letltxmacro}
\usepackage{wrapfig}
\usepackage{mathtools}
\usepackage{graphbox}
\usepackage{epigraph}
\usepackage{listings}
\usepackage{verbatim}
\usepackage{hologo}
\usepackage{multimedia}
\usepackage[absolute,overlay]{textpos}
\usepackage[euler-digits,euler-hat-accent]{eulervm}
%\logo{\pgfputat{\pgfxy(.45,.5)}{\pgfbox[center]{\includegraphics[width=1.7cm]{Figures/Bioclipse.png}}}}

\usetheme{Copenhagen}
\usecolortheme{beaver}

\definecolor{uured}{RGB}{153,0,0}
\setbeamercolor{block title}{use=structure,fg=white,bg=uured}
\setbeamercolor*{item}{fg=uured}

\newcommand{\unilogo}{
  \setlength{\TPHorizModule}{1pt}
  \setlength{\TPVertModule}{1pt}
  \begin{textblock}{1}(26,-10)
   \includegraphics[height=70pt, align=c]{Figures/uu_shadow.png}
  \end{textblock}
  } 

\pgfmathdeclarefunction{gauss}{2}{%
  \pgfmathparse{1/(#2*sqrt(2*pi))*exp(-((x-#1)^2)/(2*#2^2))}%
}
  
\makeatletter
    \newcases{mycases}{\quad}{%
        \hfil$\m@th\displaystyle{##}$}{$\m@th\displaystyle{##}$\hfil}{\lbrace}{.}
\makeatother

\addtobeamertemplate{frametitle}{}{%
    \unilogo
}

\LetLtxMacro{\oldBlock}{\block}
\LetLtxMacro{\endoldBlock}{\endblock}
\renewcommand{\block}{\begin{center}\begin{minipage}{0.8\textwidth}\oldBlock}
\renewcommand{\endblock}{\endoldBlock\end{minipage}\end{center}}
\setlength{\fboxsep}{0pt}

\begin{document}
\graphicspath{{Figures/}}
\setsansfont[ItalicFont = Optima Italic,
             BoldFont = Optima Bold,
             Ligatures=TeX ]
            {Optima Regular}
\setmainfont[ItalicFont = Optima Italic,
             BoldFont = Optima Bold,
             Ligatures=TeX]
            {Optima Regular}
%\newfontfamily\comment[]{Chalkboard}
\newfontfamily\zA[Ligatures={Common, Rare}, Variant=1] {Zapfino}
\newfontfamily\zB[Ligatures={Common, Rare}, Variant=2] {Zapfino}
\newfontfamily\zC[Ligatures={Common, Rare}, Variant=3] {Zapfino}
\newfontfamily\zD[Ligatures={Common, Rare}, Variant=4] {Zapfino}
\newfontfamily\zE[Ligatures={Common, Rare}, Variant=5] {Zapfino}
\newfontfamily\zF[Ligatures={Common, Rare}, Variant=6] {Zapfino}
\newfontfamily\zG[Ligatures={Common, Rare}, Variant=7] {Zapfino}
\renewcommand\UrlFont{\color{blue}}
\renewcommand\thefootnote{\textcolor{uured}{\arabic{footnote}}}
\setbeamercolor{alerted text}{fg=uured}
\lstset{basicstyle=\ttfamily\scriptsize, frame=single }
\newcommand{\TikZ}{{\lmr Ti\textit{k}Z}}


\title{Presentation}   
\author{Jonathan Alvarsson} 
\titlegraphic{\includegraphics[height=18pt]{Figures/pharmbio-logo-new.png}}
\date{\today} 

\setbeamertemplate{background}{%
    \parbox[c][\paperheight]{\paperwidth}{%
        \vfill
        \hfill
        \includegraphics[height=0.65\textheight]{Figures/sigill.png}
    }   
}
\begin{frame}[plain]
\unilogo \vspace{1cm} \titlepage
\begin{tikzpicture}[remember picture,overlay]
\tikz[remember picture, overlay] \fill[uured] (current page.north west) rectangle ++(\paperwidth,-0.5cm);
\end{tikzpicture}%
\end{frame}

\setbeamertemplate{background}{}
\renewcommand{\unilogo}{
  \setlength{\TPHorizModule}{1pt}
  \setlength{\TPVertModule}{1pt}
  \begin{textblock}{1}(0,0)
   \includegraphics[height=27pt, align=c]{Figures/uu.png}\includegraphics[height=10pt, align=c]{Figures/pharmbio-logo-new.png}
  \end{textblock}
  }

\section{Background}
    \begin{frame}
    \frametitle{Outline}
    \begin{minipage}{0.25\textwidth}
    \mbox{}
    \end{minipage}
    \begin{minipage}{0.6\textwidth}
    \tableofcontents
    \end{minipage}
    \end{frame}
    
\subsection{Drug discovery}
    \frame{
        \frametitle{title}
        \framesubtitle{subtitle}
        \begin{block}{Title}
        text
        \end{block}
    }

\section{Content}
    \frame{
        \frametitle{title}
        \framesubtitle{subtitle}
        \begin{oldBlock}{Title}
        text
        \end{oldBlock}
    }

\section{Conclusion}
    \frame{
        \frametitle{title}
        \framesubtitle{subtitle}
    }

\end{document}
